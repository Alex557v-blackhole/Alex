
\documentclass[12pt]{article}
\usepackage{amsmath}
\usepackage{geometry}
\geometry{margin=1in}
\title{Вазовая модель чёрной дыры: Геометрическая гипотеза внутренней Вселенной}
\author{Саша}
\date{}

\begin{document}
\maketitle

\begin{abstract}
Представлена новая интерпретация структуры чёрной дыры, основанная на вазовой геометрии. Согласно гипотезе, чёрная дыра представляет собой не точечную сингулярность, а многомерную вогнутую структуру с объективно большим внутренним объёмом.
\end{abstract}

\section*{1. Введение}
Чёрные дыры не обязаны быть точечными — они могут содержать вложенное пространство.

\section*{2. Геометрия}
\[
R(r) = r + \alpha \cdot \sin^2\left(\frac{\pi r}{L}\right)
\]

\section*{3. Метрика}
\[
ds^2 = -\left(1 - \frac{2GM}{R(r)}\right) dt^2 + \left(1 - \frac{2GM}{R(r)}\right)^{-1} dr^2 + R(r)^2 d\Omega^2
\]

\end{document}
